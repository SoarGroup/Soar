\subsection{\soarb{run}}
\label{run}
\index{run}
Begin Soar\^a��s execution cycle. 
 Status: Complete\\ 
Complete, except --output may work incorrectly due to gSKI--Jonathan 14:07, 18 Feb 2005 (EST) 
\subsubsection*{Synopsis}
\begin{verbatim}
run [count]
run -[d|e|p|o][fs] [count]
\end{verbatim}
\subsubsection*{Options}
\begin{tabular}{|l|l|}
\hline 
 -d, --decision  & Run Soar for count decision cycles.  \\
 \hline 
 -e, --elaboration  & Run Soar for count elaboration cycles.  \\
 \hline 
 -f, --forever  & Run until halted by problem-solving completion or until stopped by an interrupt.  \\
 \hline 
 -o, --output  & Run Soar until the nth time output is generated by the agent. Limited by the value of max-nil-output-cycles.  \\
 \hline 
 -p, --phase  & Run Soar by phases. A phase is either an input phase, proposal phase, decision phase, apply phase, or output phase.  \\
 \hline 
 -s, --self  & If other agents exist within the kernel, do not run them at this time.  \\
 \hline 
 count  & A single integer which specifies the number of cycles to run Soar.  \\
 \hline 
\end{tabular}
\paragraph*{Deprecated Options}
 These may be reimplemented in the future. 
\begin{tabular}{|l|l|}
\hline 
 --operator  & Run Soar until the nth time an operator is selected.  \\
 \hline 
 --state  & Run Soar until the nth time a state is selected.  \\
 \hline 
\end{tabular}
\subsubsection*{Description}
 The \textbf{run}
 command starts the Soar execution cycle or continues any execution that was temporarily stopped. The default behavior of \textbf{run}
, with no arguments, is to cause Soar to execute until it is halted or interrupted by an action of a production, or until an external interrupt is issued by the user. The \textbf{run}
 command can also specify that Soar should run only for a specific number of Soar cycles or phases (which may also be prematurely stopped by a production action or a control-C). This is helpful for debugging sessions, where users may want to pay careful attention to the specific productions that are firing and retracting. 
 The \textbf{run}
 command takes two optional arguments: an integer, \emph{count}
, which specifies how many units to run; and a \emph{units}
 flag indicating what steps or increments to use. If \emph{count}
 is specified, but no \emph{units}
 are specified, then Soar is run by decision cycles. If \emph{units}
 are specified, but \emph{count}
 is unpecified, then \emph{count}
 defaults to '1'. 
 If there are multiple Soar agents that exist in the same Soar process, then issuing a \textbf{run}
 command in any agent will cause all agents to run with the same set of parameters, unless the flag \textbf{--self}
 is specified, in which case only that agent will execute. 
\paragraph*{Note}
 If Soar has been stopped due to a \textbf{halt}
 action, an \textbf{init-soar}
 command must be issued before Soar can be restarted with the \textbf{run}
 command. 
\subsubsection*{Default Aliases}
\begin{tabular}{|l|l|}
\hline 
 Alias  & Maps to  \\
 \hline 
 d  & run -d 1  \\
 \hline 
 e  & run -e 1  \\
 \hline 
 step  & run 1  \\
 \hline 
\end{tabular}
