
\section{Soar IO Description}

\subsection{The SORTS input-link}

There are five top-level attributes on the SORTS input link, "groups", "game-info", "feature-maps", "vision-info", and "query-results". The groups, feature-maps, and vision-info structures are all part of the main visual system (see XXX), while game-info contains higher-level information about the game world, and query-results is used to communicate the results of specialized queries from Soar to the middleware.

The exact data structures are as follows:

\begin{center}
\begin{tabular}{|l|p{4.0in}|}
\hline
\multicolumn{2}{|c|}{\textbf{Attributes of io.input-link}}\\ 
\hline
attribute  &  description\\
\hline \hline
\multicolumn{2}{|l|}{\textbf{vision-info structure:}}\\ 
\hline
vision-info & Contains information on the current state of the vision system. \\
\hline
vision-info.center-x & The coordinate of the center of the region in view. \\
vision-info.center-y & \\
\hline
vision-info.focus-x & The coordinate of the center of focus (spotlight of attention). \\
vision-info.focus-y & \\
\hline
vision-info.num-objects-visible & The maximum number of objects (groups) present on the input-link. All other objects within the view window are present in feature maps. \\
\hline
vision-info.grouping-radius & All objects of the same type (except as below) and owner within this distance of each other are in the same group (set to 0 for individuals). \\
\hline
vision-info.owner-grouping & Ignore type when grouping, only group by owner (1 if enabled, 0 if disabled). \\
\hline
\multicolumn{2}{|l|}{\textbf{groups structure:}}\\ 
\hline
groups & The set of groups being attended to. \\
\hline
groups.group & Multi-valued, one instance for each group. Detailed below. \\
\hline
\multicolumn{2}{|l|}{\textbf{feature-maps structure:}}\\ 
\hline
feature-maps & Contains all feature maps- low-resolution information about certain features of unattended (but visible) objects. \\
\hline
feature-maps.friendly & Friendly feature map- each friendly unit (not group) results in one instance of this feature, marked in the sector of the group's center of gravity. \\
\hline
feature-maps.friendly-workers & A subset of the friendly feature map, showing only workers.\\
\hline
feature-maps.enemy & Similar to the above, but for enemy units.\\
\hline
feature-maps.minerals & Similar to the above, but for minerals.\\
\hline
feature-maps.moving-units & Moving units (if grouping is used, one moving object causes the whole group to be seen as moving).\\
\hline
feature-maps.(any).sector0 & Feature counts for each of the nine sectors. 0 is the upper left, 8 is the lower right.\\
.. & \\
feature-maps.(any).sector8 & \\
\hline
\multicolumn{2}{|l|}{\textbf{game-info structure:}}\\ 
\hline
game-info & General information about the game (non-visual information).\\
\hline
game-info.num-players & The number of players.\\
\hline
game-info.player-id & The ID number of the Soar player.\\
\hline
game-info.map-xdim & The dimensions of the game map. \\
game-info.map-ydim & \\
\hline
game-info.view-frame & The last view frame handled by the middleware (the number of game cycles executed).\\
\hline
game-info.my-minerals & Minerals available to the player. \\
\hline
game-info.mineral-buffer & The number of minerals to reserve for certain tasks (see section XXX).\\
\hline
game-info.worker-count & The number of units of each type owned by the player. \\
game-info.marine-count & \\
game-info.tank-count & \\
\hline
\multicolumn{2}{|l|}{\textbf{query-results structure:}}\\ 
\hline
query-results & The result of the last query to the middleware.\\
\hline
query-results.query-name & The name of the last query executed.\\
\hline
query-results.param0 & Query return values- the meaning of these is dependant on the query-name. \\
query-resutls.param1 & \\
\hline

\end{tabular}

\begin{tabular}{|l|l|p{3.5in}|}
\hline
\multicolumn{3}{|c|}{\textbf{Attributes of io.input-link.groups.group objects}}\\ 
\hline
attribute  & which groups &  description\\
\hline \hline
num-members & all & The number of individuals comprising the group. \\
\hline
type & all &The type of the group (ex: worker, mineral). \\
\hline
x-pos & all &The x,y location of the center of gravity of the group.\\
y-pos & & \\
\hline
x-min & all &The bounding box of the group.\\
x-max & & \\
y-min & & \\
y-max & & \\
\hline
health & all &The sum of the health of all units in the group.\\
\hline
taking-damage & all &The number of members of the group currently taking damage (under attack). \\
\hline
shooting & all &The number of members of the group currently attacking an enemy. \\
\hline
speed & all &The average speed of the group. \\
\hline
heading & all &The average heading of the group. \\
\hline
dist-to-focus & all &The distance from the center of gravity of the group to the attentional focus point.\\
\hline
dist-to-query & all &The distance from the center of gravity of the group to the last query location.\\
\hline
owner & all &The player number of the group's owner.\\
\hline
enemy & all &1 if the group belongs to an enemy player, 0 otherwise.\\
\hline
sticky & friendly & 1 if the group is sticky- sticky groups remain together even if they are no longer spatially close.\\
\hline
command & friendly & The last command issued to the group ("none" if no command has been issued).\\
\hline
command-running & friendly & The number of members of the group currently executing a command.\\
\hline
command-success & friendly & The number of members of the group that successfully completed the last command.\\
\hline
command-failure & friendly & The number of members of the group that unsuccessfully completed the last command.\\
\hline
minerals & friendly workers & The total number of minerals possessed by the workers in the group.\\
\hline
active-mining & friendly workers & The number of workers that are actively mining.\\
\hline
\end{tabular}
\end{center}

\subsection{The SORTS output-link}

The output-link allows the Soar agent to act in the game world by issuing commands to groups of friendly units, directly to the middleware for actions such as querying,
and adjust the parameters of the visual subsystem in the middleware. All structures on the output link are similar- the Soar agent creates an output-link.command object with a name (output-link.command.name) specified. Depending on the command name, different parameters are needed. Parameters are attached to the command structure, as in output-link.command.param0. All legal commands and their necessary parameters are detailed below.


\begin{center}
\begin{tabular}{|l|l|p{3.5in}|}
\hline
\multicolumn{3}{|c|}{\textbf{Soar output-link commands}}\\ 
\hline
command name &  parameters & description \\
\hline \hline
\multicolumn{3}{|l|}{\textbf{Vision commands}}\\ 
\hline
grouping-radius & value & Change the grouping radius of the vision system. \\
\hline
enable-owner-grouping & (none) & Enable grouping-by-owner.\\
\hline
disable-owner-grouping & (none) & Disable grouping-by-owner.\\
\hline
change-view-width & value & Change the width of the agent's view to be \textit{value}.\\
\hline 
look-at-location & x,y & Move the focus of attention to the given coordinate (which must be in the current view window.\\
\hline
move-to-location & x,y & Shift the view window to be centered at the given coordinate, and make that the focus of attention. \\
\hline
look-at-feature & feature, sector & Move the focus of attention to a given feature in a given sector. The legal features are the names of the feature-map objects on the input link.\\
\hline
move-to-feature & feature, sector & As above, but re-center the view window to the new location, also.\\
\hline
num-objects & value & Change the maximum number of objects (groups) present on the input-link at once.\\
\hline
\multicolumn{3}{|l|}{\textbf{General middleware commands}}\\ 
\hline
locate-building & building, x, y, distance & This command requests that the middleware attempt to find a location for a building of type \textit{building} approximately \textit{distance} units from the coordinate given. The resulting coordinate is returned through the query-result structure on the input-link. \\
\hline
set-mineral-buffer & value & Set the mineral buffer to \textit{value} minerals.\\
\hline
\multicolumn{3}{|l|}{\textbf{Group commands}}\\
\hline
move & group0, param0, param1 & Move the group with ID \textit{group0} to the x,y location where \textit{param0} is x and \textit{param1} is y, using the default precision of 10 (allow units to complete successfully if they are within 10 of the target).\\
\hline
move & group0, param0, param1, param2 & Move group \textit{group0} to the x,y location where \textit{param0} is x and \textit{param1} is y, using a precision of \textit{param2}.\\
\hline
build &group0, param0 .. param3 & Use group $group0$ to build a building of type \textit{param0} at the x,y location given by $param2, param3$.\\


\hline
\end{tabular}
\end{center}
