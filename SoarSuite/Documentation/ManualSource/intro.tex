% ----------------------------------------------------------------------------
\typeout{--------------- INTROduction ---------------------------------------}
\chapter{Introduction}
\label{INTRO}

Soar has been developed to be an architecture for constructing general
intelligent systems. It has been in use since 1983, and has evolved through
many different versions. This manual documents the most current of these:
Soar, version 8.5.

Our goals for Soar include that it is to be an architecture that can: \vspace{-12pt}

\begin{itemize} 
\item be used to build systems that work on the full range of tasks expected
	of an \linebreak intelligent agent, from highly routine to extremely difficult,
	open-ended problems;\vspace{-6pt}
\item represent and use appropriate forms of knowledge, such as procedural,
	declarative, episodic, and possibly iconic;\vspace{-6pt}
\item employ the full range of problem solving methods;\vspace{-6pt}
\item interact with the outside world; and\vspace{-6pt}
\item learn about all aspects of the tasks and its performance on them.
\end{itemize} 

In other words, our intention is for Soar to support all the capabilities
required of a general intelligent agent. Below are the major principles that
are the cornerstones of Soar's design:  \vspace{-12pt}

\begin{enumerate} 
\item The number of distinct architectural mechanisms should be minimized.
        In Soar there is a single representation of permanent knowledge
        (productions), a single representation of temporary knowledge (objects
        with attributes and values), a single mechanism for generating goals
        (automatic subgoaling), and a single learning mechanism (chunking).\vspace{-6pt}

\item All decisions are made through the combination of relevant knowledge at
        run-time.  In Soar, every decision is based on the current
        interpretation of sensory data and any relevant knowledge retrieved
        from permanent memory.  Decisions are never precompiled into
        uninterruptible sequences.
\end{enumerate}


% ----------------------------------------------------------------------------
% ----------------------------------------------------------------------------
\section{Using this Manual}

\nocomment{check that this describes the final form of the manual}

We expect that novice Soar users will read the manual in the order it is
presented: 

\begin{description}
\item[Chapter \ref{ARCH} and Chapter \ref{SYNTAX}] describe Soar from
different perspectives: \textbf{Chapter \ref{ARCH}} describes the Soar
architecture, but avoids issues of syntax, while \textbf{Chapter \ref{SYNTAX}}
describes the syntax of Soar, including the specific conditions and actions
allowed in Soar productions.

\item[Chapter \ref{CHUNKING}] describes chunking, Soar's learning
mechanism.  Not all users will make use of chunking, but it is important to
know that this capability exists.

\item[Chapter \ref{INTERFACE}] describes the Soar user interface --- how the
user interacts with Soar. The chapter is a catalog of user-interface commands,
grouped by functionality. For quick reference, an alphabetical listing of commands
is also provided on the back cover of the manual.
\end{description}

Advanced users will refer most often to Chapter \ref{INTERFACE}, flipping back
to Chapters \ref{ARCH} and \ref{SYNTAX} to answer specific questions.

There are several appendices included with this manual: 
\begin{description}

%\item[Appendix \ref{GLOSSARY}] is a glossary of terminology used in this manual.

\item[Appendix \ref{BLOCKSCODE}] contains an example Soar program for a simple
version of the blocks world. This blocks-world program is used as an example
throughout the manual.

%\item[Appendix \ref{USING}] is an overview of example programs currently available
%(provided with the Soar distribution) with explanations of how to run them,
%and pointers to other help sources available for novices.

%\item[Appendix \ref{DEFAULT}] describes Soar's default knowledge, which can be used
%(or not) with any Soar task.

\item[Appendix \ref{GRAMMARS}] provides a grammar for Soar productions.

\item[Appendix \ref{SUPPORT}] describes the determination of o-support.

\item[Appendix \ref{PREFERENCES}] provides a detailed explanation of the preference
resolution process.

%\item[Appendix \ref{Tcl-I/O}] gives an example of Soar I/O functions, written in Tcl.

\item[Appendix \ref{GDS}] provides an explanation of the Goal Dependency Set. 
\end{description}

\subsubsection*{Additional Back Matter}

The appendices are followed by an index; the last
pages of this manual contain a summary and index of the user-interface
functions for quick reference.


\subsubsection*{Not Described in This Manual}

Some of the more advanced features of Soar are not described in this
manual, such as how to interface with a simulator, or how to create Soar
applications using multiple interacting agents.  A discussion of
these topics is provided in a separate document, the \textit{SML Quick Start Guide}.

For novice Soar users, try \textit{The Soar 8 Tutorial}, which guides the reader 
through several example tasks and exercises.

See Section \ref{CONTACT} for information about obtaining Soar documentation.

% ----------------------------------------------------------------------------
%\section{Other Soar Documentation}
%\label{DOCUMENTATION}
%
%In addition to this manual, there are three other documents that you may want
%to obtain for more information about different aspects of Soar:
%
%\begin{description}
%\item[The Soar 8 Tutorial] is written for novice Soar users, and guides the
%	reader through several example tasks and exercises.
%\item[The Soar Advanced Applications Manual] is written for advanced Soar
%	users. This guide describes how to add input and output routines to
%	Soar programs, how to run multiple Soar ``agents'' from a single Soar
%	image, and how to extend Soar by adding your own user-interface
%	functions, simulators, or graphical user interfaces.
%\item[Soar Design Dogma] gives advice and examples about good Soar programming style. 
%        It may be helpful to both the novice and mid-level Soar user. 
%\end{description}
% ----------------------------------------------------------------------------
\section{Contacting the Soar Group}
\label{CONTACT}

\subsection*{Resources on the Internet}

The primary website for Soar is:

\hspace{2em}\soar{http://sitemaker.umich.edu/soar}.

Look here for the latest downloads, documentation, and Soar-related announcements, as well
as links to information about specific Soar research projects and researchers and a FAQ
(list of frequently asked questions) about Soar.

For questions about Soar, you may write to the Soar e-mail list at:

\hspace{2em}\soar{soar-group@lists.sourceforge.net}.

If you would like to be on this list yourself, visit:

\hspace{2em}\soar{http://lists.sourceforge.net/lists/listinfo/soar-group}.

To report Soar bugs, to check whether a bug has been reported, or to check the status
of a previously reported bug, visit:

\hspace{2em}\soar{https://winter.eecs.umich.edu/soar-bugzilla/}.


 

%The online FAQ will usually contain the most current information on Soar. It
%is available at: 

%\soar{http://acs.ist.psu.edu/soar-faq/soar-faq.html}


\subsection*{For Those Without Internet Access}

If you cannot reach us on the internet, please write to us at the following 
address:

\begin{flushleft}
\hspace{2em}The Soar Group \\
\hspace{2em}Artificial Intelligence Laboratory \\
\hspace{2em}University of Michigan\\
\hspace{2em}1101 Beal Ave.\\
\hspace{2em}Ann Arbor, MI 48109-2110 \\
\hspace{2em}USA 
\end{flushleft}

% ----------------------------------------------------------------------------
% ----------------------------------------------------------------------------
\section{A Note on Different Platforms and Operating Systems}
\label{INTRO-platforms}
\index{Unix}
\index{Linux}
\index{Macintosh}
\index{Personal Computer}
\index{Windows}
\index{Operating System}

Soar runs on a wide variety of computers, including Unix (and Linux) machines,
Macintoshes, and PCs running the Windows (95, 98, NT) operating system.

This manual documents Soar generally, although all references to files
and directories use Unix format rather than Macintosh or Windows folders.

